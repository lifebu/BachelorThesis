%% LaTeX2e class for student theses
%% sections/content.tex
%% 
%% Karlsruhe Institute of Technology
%% Institute for Program Structures and Data Organization
%% Chair for Software Design and Quality (SDQ)
%%
%% Dr.-Ing. Erik Burger
%% burger@kit.edu
%%
%% Version 1.3.5, 2020-06-26

\chapter{Motivation}
\label{ch:Motivation}

\section{Multiagent Systems}

Use Cases:
\begin{itemize}[noitemsep,nolistsep]
	\item Multiagent systems can be used in game theory and financing
	\item Reconnaissance robots covering a wide area. Communication not always possible.
	\item Smart Grid for Electricity, Power allocation, energy management.
	\item Flow Line Systems
	\item Stock markets
	\item Competitive pricing strategies
	\item Load Balancing.
	\item Network Systems (IoT).
	\item Traffic Light Control
	\item Autonomous Driving, Vehicular networks
	\item Automating turbulence modelling (aircraft design, weather forecasting, climate prediction).
	\item Control Systems for industrial processes.
	\item Intrusion Detection
	\item resource allocation for UAV Networks
	\item Large Scale City Traffic (Cityflow).
	\item Spectrum Management of cognitive radio using MARL.
\end{itemize}
Aspects:
\begin{itemize}[noitemsep,nolistsep]
	\item Ant-Colony-Optimization, which can be used for learning.
	\item Emergent Behavior.
	\item Swarm Intelligence.
	\item multi-agent reinforcement learning.
	\item multi-agent learning.
	\item game theory
\end{itemize}


%% -------------------
%% | Example content |
%% -------------------

This is the SDQ thesis template.
For more information on the formatting of theses at SDQ, please refer to
\url{https://sdqweb.ipd.kit.edu/wiki/Ausarbeitungshinweise} or to your advisor.

\section{Spacing and indentation}
To separate parts of text in \LaTeX, please use two line breaks.
They will then be set with correct indentation.
Do \emph{not} use:
\begin{itemize}
  \itemsep0em
  \item \texttt{\textbackslash\textbackslash}
  \item \texttt{\textbackslash parskip}
  \item \texttt{\textbackslash vskip}
\end{itemize} 
or other commands to manually insert spaces, since they break the layout of this template.

\section{Example: Citation}
A citation: \cite{becker2008a} 

\section{Example: Figures}
\label{sec:Introduction:Figures}
\begin{figure}
\centering
\includegraphics[width=4cm]{logos/sdqlogo}
\caption{SDQ logo}
\label{fig:sdqlogo}
\end{figure}

A reference: The SDQ logo is displayed in \autoref{fig:sdqlogo}. 
(Use \code{\textbackslash autoref\{\}} for easy referencing.) 

\section{Example: Tables}
The \texttt{booktabs} package offers nicely typeset tables, as in \autoref{tab:atable}.

\label{sec:Introduction:Tables}
\begin{table}
\centering
\begin{tabular}{r l}
\toprule
abc & def\\
ghi & jkl\\
\midrule
123 & 456\\
789 & 0AB\\
\bottomrule
\end{tabular}
\caption{A table}
\label{tab:atable}
\end{table}

\section{Example: Formula}
One of the nice things about the Linux Libertine font is that it comes with
a math mode package.
\begin{displaymath}
f(x)=\Omega(g(x))\ (x\rightarrow\infty)\;\Leftrightarrow\;
\limsup_{x \to \infty} \left|\frac{f(x)}{g(x)}\right|> 0
\end{displaymath}

%% --------------------
%% | /Example content |
%% --------------------