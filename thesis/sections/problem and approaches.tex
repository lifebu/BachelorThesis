%% LaTeX2e class for student theses
%% sections/content.tex
%% 
%% Karlsruhe Institute of Technology
%% Institute for Program Structures and Data Organization
%% Chair for Software Design and Quality (SDQ)
%%
%% Dr.-Ing. Erik Burger
%% burger@kit.edu
%%
%% Version 1.3.5, 2020-06-26

\chapter{Problem and Approaches}
\label{ch:ProblemApproaches}

Agent:
\begin{itemize}[noitemsep,nolistsep]
    \item performance measure: which one?
    \item needs information gathering
    \item uses percepts to find correct action from prior knowledge
    \item communication is important
    \item learning while a game is running?
\end{itemize}
\medskip
Multiagent:
\begin{itemize}[noitemsep,nolistsep]
    \item multiagent (decentralized planning with coordination) or multibody (centralized planning)?
    \item multieffector (units can walk and attack) (can be made easier).
\end{itemize}
\medskip
Environment:
\begin{itemize}[noitemsep,nolistsep]
    \item partially observable (fog of war).
    \item cooperative multiagent environment for a given player.
    \item competitive multiagent environment between players.
    \item usually deterministic, games may use RNG than it would stochastic.
    \item stochastic because of partially observability.
    \item sequentiell  (current actions depend on previous actions).
    \item static environment (when AI is framelocked), otherwise dynamic
    \item pseudo-continious environment
    \item known, but partially observable.
    \item basically hardest case: partially observable, multiagent, stochastic, sequentiell, dynamic, continuous and unknown.
\end{itemize}

